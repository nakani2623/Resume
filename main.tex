\documentclass[10pt, letterpaper]{article}

% Packages:
\usepackage[
    ignoreheadfoot, % set margins without considering header and footer
    top=2 cm, % seperation between body and page edge from the top
    bottom=2 cm, % seperation between body and page edge from the bottom
    left=2 cm, % seperation between body and page edge from the left
    right=2 cm, % seperation between body and page edge from the right
    footskip=1.0 cm, % seperation between body and footer
    % showframe % for debugging 
]{geometry} % for adjusting page geometry
\usepackage{titlesec} % for customizing section titles
\usepackage{tabularx} % for making tables with fixed width columns
\usepackage{array} % tabularx requires this
\usepackage[dvipsnames]{xcolor} % for coloring text
\definecolor{primaryColor}{RGB}{0, 0, 0} % define primary color
\usepackage{enumitem} % for customizing lists
\usepackage{fontawesome5} % for using icons
\usepackage{amsmath} % for math
\usepackage[
    pdftitle={John Doe's CV},
    pdfauthor={John Doe},
    pdfcreator={LaTeX with RenderCV},
    colorlinks=true,
    urlcolor=primaryColor
]{hyperref} % for links, metadata and bookmarks
\usepackage[pscoord]{eso-pic} % for floating text on the page
\usepackage{calc} % for calculating lengths
\usepackage{bookmark} % for bookmarks
\usepackage{lastpage} % for getting the total number of pages
\usepackage{changepage} % for one column entries (adjustwidth environment)
\usepackage{paracol} % for two and three column entries
\usepackage{ifthen} % for conditional statements
\usepackage{needspace} % for avoiding page brake right after the section title
\usepackage{iftex} % check if engine is pdflatex, xetex or luatex

% Ensure that generate pdf is machine readable/ATS parsable:
\ifPDFTeX
    \input{glyphtounicode}
    \pdfgentounicode=1
    \usepackage[T1]{fontenc}
    \usepackage[utf8]{inputenc}
    \usepackage{lmodern}
\fi

\usepackage{charter}

% Some settings:
\raggedright
\AtBeginEnvironment{adjustwidth}{\partopsep0pt} % remove space before adjustwidth environment
\pagestyle{empty} % no header or footer
\setcounter{secnumdepth}{0} % no section numbering
\setlength{\parindent}{0pt} % no indentation
\setlength{\topskip}{0pt} % no top skip
\setlength{\columnsep}{0.15cm} % set column seperation
\pagenumbering{gobble} % no page numbering

\titleformat{\section}{\needspace{4\baselineskip}\bfseries\large}{}{0pt}{}[\vspace{1pt}\titlerule]

\titlespacing{\section}{
    % left space:
    -1pt
}{
    % top space:
    0.3 cm
}{
    % bottom space:
    0.2 cm
} % section title spacing

\renewcommand\labelitemi{$\vcenter{\hbox{\small$\bullet$}}$} % custom bullet points
\newenvironment{highlights}{
    \begin{itemize}[
        topsep=0.10 cm,
        parsep=0.10 cm,
        partopsep=0pt,
        itemsep=0pt,
        leftmargin=0 cm + 10pt
    ]
}{
    \end{itemize}
} % new environment for highlights


\newenvironment{highlightsforbulletentries}{
    \begin{itemize}[
        topsep=0.10 cm,
        parsep=0.10 cm,
        partopsep=0pt,
        itemsep=0pt,
        leftmargin=10pt
    ]
}{
    \end{itemize}
} % new environment for highlights for bullet entries

\newenvironment{onecolentry}{
    \begin{adjustwidth}{
        0 cm + 0.00001 cm
    }{
        0 cm + 0.00001 cm
    }
}{
    \end{adjustwidth}
} % new environment for one column entries

\newenvironment{twocolentry}[2][]{
    \onecolentry
    \def\secondColumn{#2}
    \setcolumnwidth{\fill, 4.5 cm}
    \begin{paracol}{2}
}{
    \switchcolumn \raggedleft \secondColumn
    \end{paracol}
    \endonecolentry
} % new environment for two column entries

\newenvironment{threecolentry}[3][]{
    \onecolentry
    \def\thirdColumn{#3}
    \setcolumnwidth{, \fill, 4.5 cm}
    \begin{paracol}{3}
    {\raggedright #2} \switchcolumn
}{
    \switchcolumn \raggedleft \thirdColumn
    \end{paracol}
    \endonecolentry
} % new environment for three column entries

\newenvironment{header}{
    \setlength{\topsep}{0pt}\par\kern\topsep\centering\linespread{1.5}
}{
    \par\kern\topsep
} % new environment for the header

\newcommand{\placelastupdatedtext}{% \placetextbox{<horizontal pos>}{<vertical pos>}{<stuff>}
  \AddToShipoutPictureFG*{% Add <stuff> to current page foreground
    \put(
        \LenToUnit{\paperwidth-2 cm-0 cm+0.05cm},
        \LenToUnit{\paperheight-1.0 cm}
    ){\vtop{{\null}\makebox[0pt][c]{
        \small\color{gray}\textit{Last updated in September 2024}\hspace{\widthof{Last updated in September 2024}}
    }}}%
  }%
}%

% save the original href command in a new command:
\let\hrefWithoutArrow\href

% new command for external links:


\begin{document}
    \newcommand{\AND}{\unskip
        \cleaders\copy\ANDbox\hskip\wd\ANDbox
        \ignorespaces
    }
    \newsavebox\ANDbox
    \sbox\ANDbox{$|$}

    \begin{header}
        % TODO: name
        \fontsize{25 pt}{25 pt}\selectfont Zhongren Sun

        \vspace{5 pt}

        \normalsize
        \mbox{Forest Lodge, NSW}%
        \kern 5.0 pt%
        \AND%
        \kern 5.0 pt%
        \mbox{\hrefWithoutArrow{mailto:nakani2623@outlook.com}{nakani2623@outlook.com}}%
        \kern 5.0 pt%
        \AND%
        \kern 5.0 pt%
        \mbox{\hrefWithoutArrow{tel:+61 402 750 770}{0402 750 770}}%
        \kern 5.0 pt%
        \AND%
        \kern 5.0 pt%
        \mbox{\hrefWithoutArrow{https://github.com/nakani2623}{github.com/nakani2623}}%
    \end{header}

    \vspace{5 pt - 0.3 cm}

    \section{Education}
        \begin{twocolentry}{
            Mar 2021 - Nov 2024
        }
            \textbf{University of Sydney}, Bachelor in Computer Science
        \end{twocolentry}
        \vspace{0.10 cm}

        \begin{onecolentry}
            \begin{highlights}
                \item \textbf{Coursework:} Data structures and Algorithms, Operating Systems, Computational Theory
            \end{highlights}
        \end{onecolentry}

        \begin{twocolentry}{
            Feb 2019 - Nov 2020
        }
        \textbf{Hobart College}
        \end{twocolentry}

    
    \section{Experience}
        \begin{twocolentry}{
            Jul 2024 – Nov 2024
        }
            \textbf{Software Engineer}, Asiga -- Sydney, NSW
        \end{twocolentry}

        \vspace{0.10 cm}
        \begin{onecolentry}
            \begin{highlights}
                \item developed an in-fill algorithm, generating vector paths of 2 types (model \& support) in the context of multijet modeling 3d printers. Generator application in C++. Develop a visualization tool to demonstrate the functionality.
                \item Responsible for system architecture design, algorithm design, programming.
                \item Sprint with Extreme Programming
                \item Develop with: C++, Git, CMake
            \end{highlights}
        \end{onecolentry}

    \section{Projects}
        \begin{twocolentry}{tbc..}
            \textbf{Vending Machine Software: Lite Snacks}
        \end{twocolentry}
        \vspace{0.10 cm}
        \begin{onecolentry}
            \begin{highlights}
                \item Developed a vending machine software using Scrum method in a team of 4.
                \item Responsible for writing and maintaining the product backlog
                \item Integrates team members' code of the current day
                \item Develop with: Java, Git, Gradle
            \end{highlights}
        \end{onecolentry} 
        \vspace{0.2 cm}

        \begin{twocolentry}{tbc..}
            \textbf{Sydney Livability Analysis}
        \end{twocolentry}
        \vspace{0.10 cm}
        \begin{onecolentry}
            \begin{highlights}
                \item Analyze the livability of each SA2 block in Sydney and its surrounding areas
                \item Select livable factor data for the target group, use pandas and numpy to clean the data, and use PostgreSQL to build a database. The database integration includes various data types (e.g. geom) .  helped team members solving the problem of types of the shape data and make all geometric data compatible in the database.
            \end{highlights}
        \end{onecolentry}
        \vspace{0.2 cm}

        \begin{twocolentry}{tbc..}
            \textbf{Virtual Machine}
        \end{twocolentry}
        \vspace{0.10 cm}
        \begin{onecolentry}
            \begin{highlights}
                \item todo
            \end{highlights}
        \end{onecolentry}
        \vspace{0.2 cm}

        \begin{twocolentry}{
            \href{https://github.com/sinaatalay/rendercv}{github.com/name/repo}
        }
        \textbf{Synchronised communication}\end{twocolentry}
        \vspace{0.10 cm}
        \begin{onecolentry}
            \begin{highlights}
                \item Simulating communication between multiple objects
                \item Tools Used: C/C++
            \end{highlights}
        \end{onecolentry}
        \vspace{0.2 cm}

        \begin{twocolentry}{tbc..}
            \textbf{Multi Level Queue Dispatcher}
        \end{twocolentry}
        \vspace{0.10 cm}
        \begin{onecolentry}
            \begin{highlights}
                \item Implemented a Multi-Level Queue Dispatcher for a uniprocessor system 
                \item using: C
            \end{highlights}
        \end{onecolentry} 
        \vspace{0.2 cm}



    \section{Skills}
        \begin{onecolentry}
            \textbf{Languages:} C/C++, Java, Python
        \end{onecolentry}

        \vspace{0.2 cm}

        \begin{onecolentry}
            \textbf{Algorithms:} Implement algorithms with Greedy, D\&C, DP, Reduction approach
        \end{onecolentry}
        \vspace{0.2 cm}

        \begin{onecolentry}
            \textbf{Software Development:} Agile method, XP
        \end{onecolentry}
        \vspace{0.2 cm}

        \begin{onecolentry}
            \textbf{Data Science:} Data processing with Python, R (using pandas, numpy, tidyverse) 
        \end{onecolentry}
        \vspace{0.2 cm}

        \begin{onecolentry}
            \textbf{Languages:} EAL/D
        \end{onecolentry}
        \vspace{0.2 cm}

        \begin{onecolentry}
            \textbf{Others:} Git, Microsoft Office, basic LaTeX
        \end{onecolentry}

\end{document}